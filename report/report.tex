% #######################################
% ########### FILL THESE IN #############
% #######################################
\def\mytitle{Coursework 2 Report}
\def\mykeywords{Python, Flask, Web Development, HTML, CSS}
\def\myauthor{Callum Carnegie}
\def\contact{40323323@napier.ac.uk}
\def\mymodule{Advanced Web Technologies (SET09103)}
% #######################################
% #### YOU DON'T NEED TO TOUCH BELOW ####
% #######################################
\documentclass[10pt, a4paper]{article}
\usepackage[a4paper,outer=1.5cm,inner=1.5cm,top=1.75cm,bottom=1.5cm]{geometry}
\twocolumn
\usepackage{graphicx}
\graphicspath{{./images/}}
%colour our links, remove weird boxes
\usepackage[colorlinks,linkcolor={black},citecolor={blue!80!black},urlcolor={blue!80!black}]{hyperref}
%Stop indentation on new paragraphs
\usepackage[parfill]{parskip}
%% Arial-like font
\IfFileExists{uarial.sty}
{
	\usepackage[english]{babel}
	\usepackage[T1]{fontenc}
	\usepackage{uarial}
	\renewcommand{\familydefault}{\sfdefault}
}{
	\GenericError{}{Couldn't find Arial font}{ you may need to install 'nonfree' fonts on your system}{}
	\usepackage{lmodern}
	\renewcommand*\familydefault{\sfdefault}
}
%Napier logo top right
\usepackage{watermark}
%Lorem Ipusm dolor please don't leave any in you final report ;)
\usepackage{lipsum}
\usepackage{xcolor}
\usepackage{listings}
%give us the Capital H that we all know and love
\usepackage{float}
%tone down the line spacing after section titles
\usepackage{titlesec}
%Cool maths printing
\usepackage{amsmath}
%PseudoCode
\usepackage{algorithm2e}

\titlespacing{\subsection}{0pt}{\parskip}{-3pt}
\titlespacing{\subsubsection}{0pt}{\parskip}{-\parskip}
\titlespacing{\paragraph}{0pt}{\parskip}{\parskip}
\newcommand{\figuremacro}[5]{
	\begin{figure}[#1]
		\centering
		\includegraphics[width=#5\columnwidth]{#2}
		\caption[#3]{\textbf{#3}#4}
		\label{fig:#2}
	\end{figure}
}

\lstset{
	escapeinside={/*@}{@*/}, language=Python,
	basicstyle=\fontsize{8.5}{12}\selectfont,
	numbers=left,numbersep=2pt,xleftmargin=2pt,frame=tb,
	columns=fullflexible,showstringspaces=false,tabsize=4,
	keepspaces=true,showtabs=false,showspaces=false,
	backgroundcolor=\color{white},morekeywords={with,as},captionpos=t,
	lineskip=-0.4em,aboveskip=10pt,extendedchars=true,breaklines=true,
	prebreak=\raisebox{0ex}[0ex][0ex]{\ensuremath{\hookleftarrow}},
	keywordstyle=\color[rgb]{0,0,1},
	commentstyle=\color[rgb]{0.133,0.545,0.133},
	stringstyle=\color[rgb]{0.627,0.126,0.941}
}

\thiswatermark{\centering \put(336.5,-38.0){\includegraphics[scale=0.8]{logo}} }
\title{\mytitle}
\author{\myauthor\hspace{1em}\\\contact\\Edinburgh Napier University\hspace{0.5em}-\hspace{0.5em}\mymodule}
\date{}
\hypersetup{pdfauthor=\myauthor,pdftitle=\mytitle,pdfkeywords=\mykeywords}
\sloppy
% #######################################
% ########### START FROM HERE ###########
% #######################################
\begin{document}
	\maketitle
	
	\section{Introduction}
	
	
	\section{Design}
	\subsection{Front-End}
	When designing the application various decisions were made to keep the accessibility suitable for a wide variety of users from different ages and technological capabilities. The colour scheme was monochrome and selected with contrast in mind as to not make it difficult for viewers with limited eyesight and make it as clear as possible. The monochrome would also help with colour blindness as there are little colour significance. The main colour used was blue, as it provides a calming and friendly atmosphere for the user. The colours used with parts of the application that provide functionality buttons are coloured in a way that stands out, and can separate the importance of elements in the page.
	
	The font used throughout the application was a sans-serif font that is clear at both small and larger sizes, while also giving a more friendly atmosphere. The layouts used were made in a way that would be most clear, and the content was separated in a way that brings attention to the most important parts so when viewed it makes more sense.
	
	\subsection{Back-End}
	When creating the code for the majority of the project Python 3.6 was used, but compatibility with earlier versions of Python 2 and 3 were kept in mind and attempted to be kept intact. The only changes was simple things such as using parenthesis on print and using the from \texttt{\_\_from future\_\_} import when debugging with print statements. The database used was Sqlite3 which goes well with python with little to be added.
	
	
	\section{Enhancements}
	An enhancement I would add is to make the login correctly work with the posting. As currently it requires the author to be added manually. Ideally the author would come from the user's username from the database. It could possibly even link to a user's page.
	
	Another enhancement would be to add commenting functionality and voting. These were two features that I would have added to make a more community based feel to the application. This would allow for discussion between users on the posts, where now they would have to make another post to comment on things causing a lot of not organised posting.
	
	
	\section{Critical Evaluation}
	For this app I could have implemented an Admin role and utilities, but I felt that it was Not necessary to demonstrate the basic idea of the application. It could be implemented at a later date for deleting or updating data possibly.
	
	\section{Personal Evaluation}
	During the project the implementation and design of the code was done to what I believe was up to a high standard and to my best capabilities.
	
	Some of the time management left plenty to be desired although, this would have made the whole project a much more calm and less stressful experience. Especially with the report, as I feel that writing small parts as I went along could have really helped add more and made it easier.
\end{document}